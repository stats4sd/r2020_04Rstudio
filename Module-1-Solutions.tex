% Options for packages loaded elsewhere
\PassOptionsToPackage{unicode}{hyperref}
\PassOptionsToPackage{hyphens}{url}
%
\documentclass[
]{article}
\usepackage{lmodern}
\usepackage{amssymb,amsmath}
\usepackage{ifxetex,ifluatex}
\ifnum 0\ifxetex 1\fi\ifluatex 1\fi=0 % if pdftex
  \usepackage[T1]{fontenc}
  \usepackage[utf8]{inputenc}
  \usepackage{textcomp} % provide euro and other symbols
\else % if luatex or xetex
  \usepackage{unicode-math}
  \defaultfontfeatures{Scale=MatchLowercase}
  \defaultfontfeatures[\rmfamily]{Ligatures=TeX,Scale=1}
\fi
% Use upquote if available, for straight quotes in verbatim environments
\IfFileExists{upquote.sty}{\usepackage{upquote}}{}
\IfFileExists{microtype.sty}{% use microtype if available
  \usepackage[]{microtype}
  \UseMicrotypeSet[protrusion]{basicmath} % disable protrusion for tt fonts
}{}
\makeatletter
\@ifundefined{KOMAClassName}{% if non-KOMA class
  \IfFileExists{parskip.sty}{%
    \usepackage{parskip}
  }{% else
    \setlength{\parindent}{0pt}
    \setlength{\parskip}{6pt plus 2pt minus 1pt}}
}{% if KOMA class
  \KOMAoptions{parskip=half}}
\makeatother
\usepackage{xcolor}
\IfFileExists{xurl.sty}{\usepackage{xurl}}{} % add URL line breaks if available
\IfFileExists{bookmark.sty}{\usepackage{bookmark}}{\usepackage{hyperref}}
\hypersetup{
  hidelinks,
  pdfcreator={LaTeX via pandoc}}
\urlstyle{same} % disable monospaced font for URLs
\usepackage[margin=1in]{geometry}
\usepackage{color}
\usepackage{fancyvrb}
\newcommand{\VerbBar}{|}
\newcommand{\VERB}{\Verb[commandchars=\\\{\}]}
\DefineVerbatimEnvironment{Highlighting}{Verbatim}{commandchars=\\\{\}}
% Add ',fontsize=\small' for more characters per line
\usepackage{framed}
\definecolor{shadecolor}{RGB}{248,248,248}
\newenvironment{Shaded}{\begin{snugshade}}{\end{snugshade}}
\newcommand{\AlertTok}[1]{\textcolor[rgb]{0.94,0.16,0.16}{#1}}
\newcommand{\AnnotationTok}[1]{\textcolor[rgb]{0.56,0.35,0.01}{\textbf{\textit{#1}}}}
\newcommand{\AttributeTok}[1]{\textcolor[rgb]{0.77,0.63,0.00}{#1}}
\newcommand{\BaseNTok}[1]{\textcolor[rgb]{0.00,0.00,0.81}{#1}}
\newcommand{\BuiltInTok}[1]{#1}
\newcommand{\CharTok}[1]{\textcolor[rgb]{0.31,0.60,0.02}{#1}}
\newcommand{\CommentTok}[1]{\textcolor[rgb]{0.56,0.35,0.01}{\textit{#1}}}
\newcommand{\CommentVarTok}[1]{\textcolor[rgb]{0.56,0.35,0.01}{\textbf{\textit{#1}}}}
\newcommand{\ConstantTok}[1]{\textcolor[rgb]{0.00,0.00,0.00}{#1}}
\newcommand{\ControlFlowTok}[1]{\textcolor[rgb]{0.13,0.29,0.53}{\textbf{#1}}}
\newcommand{\DataTypeTok}[1]{\textcolor[rgb]{0.13,0.29,0.53}{#1}}
\newcommand{\DecValTok}[1]{\textcolor[rgb]{0.00,0.00,0.81}{#1}}
\newcommand{\DocumentationTok}[1]{\textcolor[rgb]{0.56,0.35,0.01}{\textbf{\textit{#1}}}}
\newcommand{\ErrorTok}[1]{\textcolor[rgb]{0.64,0.00,0.00}{\textbf{#1}}}
\newcommand{\ExtensionTok}[1]{#1}
\newcommand{\FloatTok}[1]{\textcolor[rgb]{0.00,0.00,0.81}{#1}}
\newcommand{\FunctionTok}[1]{\textcolor[rgb]{0.00,0.00,0.00}{#1}}
\newcommand{\ImportTok}[1]{#1}
\newcommand{\InformationTok}[1]{\textcolor[rgb]{0.56,0.35,0.01}{\textbf{\textit{#1}}}}
\newcommand{\KeywordTok}[1]{\textcolor[rgb]{0.13,0.29,0.53}{\textbf{#1}}}
\newcommand{\NormalTok}[1]{#1}
\newcommand{\OperatorTok}[1]{\textcolor[rgb]{0.81,0.36,0.00}{\textbf{#1}}}
\newcommand{\OtherTok}[1]{\textcolor[rgb]{0.56,0.35,0.01}{#1}}
\newcommand{\PreprocessorTok}[1]{\textcolor[rgb]{0.56,0.35,0.01}{\textit{#1}}}
\newcommand{\RegionMarkerTok}[1]{#1}
\newcommand{\SpecialCharTok}[1]{\textcolor[rgb]{0.00,0.00,0.00}{#1}}
\newcommand{\SpecialStringTok}[1]{\textcolor[rgb]{0.31,0.60,0.02}{#1}}
\newcommand{\StringTok}[1]{\textcolor[rgb]{0.31,0.60,0.02}{#1}}
\newcommand{\VariableTok}[1]{\textcolor[rgb]{0.00,0.00,0.00}{#1}}
\newcommand{\VerbatimStringTok}[1]{\textcolor[rgb]{0.31,0.60,0.02}{#1}}
\newcommand{\WarningTok}[1]{\textcolor[rgb]{0.56,0.35,0.01}{\textbf{\textit{#1}}}}
\usepackage{graphicx,grffile}
\makeatletter
\def\maxwidth{\ifdim\Gin@nat@width>\linewidth\linewidth\else\Gin@nat@width\fi}
\def\maxheight{\ifdim\Gin@nat@height>\textheight\textheight\else\Gin@nat@height\fi}
\makeatother
% Scale images if necessary, so that they will not overflow the page
% margins by default, and it is still possible to overwrite the defaults
% using explicit options in \includegraphics[width, height, ...]{}
\setkeys{Gin}{width=\maxwidth,height=\maxheight,keepaspectratio}
% Set default figure placement to htbp
\makeatletter
\def\fps@figure{htbp}
\makeatother
\setlength{\emergencystretch}{3em} % prevent overfull lines
\providecommand{\tightlist}{%
  \setlength{\itemsep}{0pt}\setlength{\parskip}{0pt}}
\setcounter{secnumdepth}{-\maxdimen} % remove section numbering

\author{}
\date{\vspace{-2.5em}}

\begin{document}

\hypertarget{module-1-exercises}{%
\subsection{Module 1: Exercises}\label{module-1-exercises}}

(Note - these solutions contain embedded latex code for creating the
mathematical formulae. Whenever you see the \$ \$ symbols in the main
text this denotes some latex code. If you hover over it you will see the
formula displayed as it would appear when compiled. So dont worry if
this looks odd - this is not R code, and you can learn latex on another
course!)

\hypertarget{question-1}{%
\subsubsection{Question 1:}\label{question-1}}

I am trying to use R as a calculator, and am receiving an error. Can you
spot my error and update the code?

(42-5/(88+6)

\hypertarget{answer-1}{%
\subsubsection{Answer 1:}\label{answer-1}}

My brackets are incorrectly specified as I open two sets of brackets but
only close one. The correct solution is probably either
\texttt{(42-5)/(88+6)} or \texttt{42-(5/(88+6))}, depending on what I
actually wanted to do The former is probably more likely!

\begin{Shaded}
\begin{Highlighting}[]
\NormalTok{(}\DecValTok{42-5}\NormalTok{)}\OperatorTok{/}\NormalTok{(}\DecValTok{88}\OperatorTok{+}\DecValTok{6}\NormalTok{)}
\end{Highlighting}
\end{Shaded}

\begin{verbatim}
## [1] 0.393617
\end{verbatim}

\hypertarget{question-2}{%
\subsubsection{Question 2}\label{question-2}}

This question will use the same earthquakes dataset from the quiz,
showing the magnitude of earthquakes occuring in the ocean around Fiji
since 1964, as well as the number of different stations reporting the
earthquake. This has been loaded into the R sessions as a data frame
called \texttt{quakes}

\hypertarget{question-2a}{%
\subsubsection{Question 2a}\label{question-2a}}

Write a command to determine the largest magnitude (\texttt{mag})
earthquake recorded?

\hypertarget{answer-2a}{%
\subsubsection{Answer 2a}\label{answer-2a}}

I would need to use the \texttt{max} command, and then specify that i
want to use the \texttt{mag} column within the \texttt{quakes} dataset
by using the data frame name \texttt{quakes} followed by a \texttt{\$}
follwed by the column name \texttt{mag}

\begin{Shaded}
\begin{Highlighting}[]
\KeywordTok{max}\NormalTok{(quakes}\OperatorTok{$}\NormalTok{mag)}
\end{Highlighting}
\end{Shaded}

\begin{verbatim}
## [1] 6.4
\end{verbatim}

\hypertarget{question-2b}{%
\subsubsection{Question 2b}\label{question-2b}}

Write a command to determine the smallest depth (\texttt{depth}) below
surface that an earthquake was recorded?

\hypertarget{answer-2b}{%
\subsubsection{Answer 2b}\label{answer-2b}}

I would need to use the \texttt{min} command, and then specify that i
want to use the \texttt{depth} column within the \texttt{quakes} dataset
by using the data frame name \texttt{quakes} followed by a \texttt{\$}
follwed by the column name \texttt{depth}

\begin{Shaded}
\begin{Highlighting}[]
\KeywordTok{min}\NormalTok{(quakes}\OperatorTok{$}\NormalTok{mag)}
\end{Highlighting}
\end{Shaded}

\begin{verbatim}
## [1] 4
\end{verbatim}

\hypertarget{question-2c}{%
\subsubsection{Question 2c}\label{question-2c}}

I would like to obtain the standard deviation of the earthquake
magnitude column from the \texttt{quakes} dataset. See if you can find
the function for standard deviation in R (we have not mentioned it in
the course workbook so far) and then apply it to the relevant column

\hypertarget{answer-2c}{%
\subsubsection{Answer 2c}\label{answer-2c}}

From a bit of searching, hopefully you found the \texttt{sd} function.
This works in the same way as we have seen with \texttt{max} and
\texttt{min}

\begin{Shaded}
\begin{Highlighting}[]
\KeywordTok{sd}\NormalTok{(quakes}\OperatorTok{$}\NormalTok{mag)}
\end{Highlighting}
\end{Shaded}

\begin{verbatim}
## [1] 0.402773
\end{verbatim}

\hypertarget{question-3}{%
\subsubsection{Question 3}\label{question-3}}

I am a Physicist who is doing some equations involving projectile motion
and want to write some code to save me some time. The formula for this
is \(Range = \frac{v^{2}} {g} \times sin(2 \theta)\)

Create objects called \texttt{v},\texttt{g} and \texttt{theta} and set v
equal to 5; g equal to 9.8 and theta equal to 45.

Then, you will need to write some code to convert an angle in degrees to
an angle in radians, as trigonometric functions in R only operate in
radians. So you will need to multiply theta by \(\frac{\pi} {180}\).
Store this as an object called \texttt{theta\_radian}. The constant
\(\pi\) exists already in R as a named object called \texttt{pi}.

Then write the code to calculate the range I would expect my object to
be projected if initial velocity was 5\(m/s\), the angle of projection
was 45 degrees and we were on Earth, so the gravity was equal to
9.8\(m/s^2\)

\#Answer 3

I first will need to create constants using the assign operator
\texttt{\textless{}-} as specified in the question.

Then write the transformation line to convert theta to radians.

Then I can use these objects in the formula. Make sure you are careful
with use of brackets. Remember to use the converted object
\texttt{theta\_radian} rather than the original object \texttt{theta}.
Also remember that you will need to write \texttt{sin(2*theta\_radian)}
to multiple theta by 2. R will not be able to interpret
\texttt{sin(2theta\_radian)}

\begin{Shaded}
\begin{Highlighting}[]
\NormalTok{v<-}\DecValTok{5}
\NormalTok{g<-}\FloatTok{9.8}
\NormalTok{theta<-}\DecValTok{45}

\NormalTok{theta_radian<-theta}\OperatorTok{*}\NormalTok{pi}\OperatorTok{/}\DecValTok{180}


\NormalTok{(v}\OperatorTok{^}\DecValTok{2} \OperatorTok{/}\StringTok{ }\NormalTok{g)}\OperatorTok{*}\KeywordTok{sin}\NormalTok{(}\DecValTok{2}\OperatorTok{*}\NormalTok{theta_radian)}
\end{Highlighting}
\end{Shaded}

\begin{verbatim}
## [1] 2.55102
\end{verbatim}

Now I want to see what happens when I modify the angle. Instead of a
single number, replace theta with several numbers to see how the range
varies - I am interested in seeing the results for angles of 30, 35, 40,
45 and 50 degrees.

\#\#Answer 3b

We will use the \texttt{c()} to provide multiple angles to the theta
object. We do not need to change any other part of the code other than
theta.

\begin{Shaded}
\begin{Highlighting}[]
\NormalTok{v<-}\DecValTok{5}
\NormalTok{g<-}\FloatTok{9.8}
\NormalTok{theta<-}\KeywordTok{c}\NormalTok{(}\DecValTok{30}\NormalTok{,}\DecValTok{35}\NormalTok{,}\DecValTok{40}\NormalTok{,}\DecValTok{45}\NormalTok{,}\DecValTok{50}\NormalTok{)}

\NormalTok{theta_radian<-theta}\OperatorTok{*}\NormalTok{pi}\OperatorTok{/}\DecValTok{180}


\NormalTok{(v}\OperatorTok{^}\DecValTok{2} \OperatorTok{/}\StringTok{ }\NormalTok{g)}\OperatorTok{*}\KeywordTok{sin}\NormalTok{(}\DecValTok{2}\OperatorTok{*}\NormalTok{theta_radian)}
\end{Highlighting}
\end{Shaded}

\begin{verbatim}
## [1] 2.209248 2.397175 2.512265 2.551020 2.512265
\end{verbatim}

You should see that the distance is furthest for an angle of 45 degrees.
From my limited knowledge of physics, this sounds like it makes sense to
me!

\hypertarget{question-4}{%
\subsubsection{Question 4:}\label{question-4}}

A task I am sure you remember from school is to solve a quadratic
equation using the formula \(x=\frac{-b\pm\sqrt{b^2-4ac}}{2a}\). Write
some R code to find the two values of x when \(x^2-9x+19=0\) . As a
reminder, in the quadratic equation formula from this particular example
\texttt{a} would be 1; \texttt{b} would be -9 and \texttt{c} would be 19

\#\#\#Answer 4

I am going to write this in general terms, similar to question 3, so
that if i wanted to change my code for a different formula later, then i
could. However, I didn't explicitly ask you to do that, so if you
directly plugged in 1, -9 and 19 into a formula this would still be
fine.

The main trick here is to use the \texttt{c()} function to replace the
+- operator as this is equivalent to saying ``plus one times'' or
``minus one times'', so we can provide a vector of \texttt{-1} and
\texttt{1} to the code we write so that we only have to write one line
for the main part of the solution.

Brackets are incredibly easy to get wrong on this one. Be extremely
careful working out where to place them! My solution below has the
minimum number of brackets necessary (due to BODMAS rules), but there is
no harm in including extra brackets, just to be safe, if you want to
ensure that the order of operations acts as you expect.

As with question 3 - make sure to write \texttt{4*a*c} to multiply these
together; R would not be able to interpret \texttt{4ac}. However R can
interpet \texttt{-b} providing b is a number. If you wrote \texttt{-1*b}
here though that would also be perfectly correct.

\begin{Shaded}
\begin{Highlighting}[]
\NormalTok{a<-}\DecValTok{1}
\NormalTok{b<-}\OperatorTok{-}\DecValTok{9}
\NormalTok{c<-}\DecValTok{19}

\NormalTok{x<-(}\OperatorTok{-}\NormalTok{b}\OperatorTok{+}\KeywordTok{c}\NormalTok{(}\DecValTok{1}\NormalTok{,}\OperatorTok{-}\DecValTok{1}\NormalTok{)}\OperatorTok{*}\KeywordTok{sqrt}\NormalTok{(b}\OperatorTok{^}\DecValTok{2-4}\OperatorTok{*}\NormalTok{a}\OperatorTok{*}\NormalTok{c))}\OperatorTok{/}\NormalTok{(}\DecValTok{2}\OperatorTok{*}\NormalTok{a)}

\NormalTok{x}
\end{Highlighting}
\end{Shaded}

\begin{verbatim}
## [1] 5.618034 3.381966
\end{verbatim}

As i'm sure you learnt at school - it's always good practice to plug
these numbers back into the equation to see if it makes sense! Saving my
object as x in the previous step makes this easy.

\begin{Shaded}
\begin{Highlighting}[]
\NormalTok{x}\OperatorTok{^}\DecValTok{2} \OperatorTok{-}\NormalTok{(}\DecValTok{9}\OperatorTok{*}\NormalTok{x) }\OperatorTok{+}\DecValTok{19}
\end{Highlighting}
\end{Shaded}

\begin{verbatim}
## [1] 3.552714e-15 0.000000e+00
\end{verbatim}

You will see here that you don't actually get a zero for the first
number. Unfortunately R will sometimes come up with rounding errors.
Remember in scientific notation that ``3.552714e-15'' is equal to
``0.00000000000000355''.So i think i am happy to conclude that I got my
formula correct!

\end{document}
